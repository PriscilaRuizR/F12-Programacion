\documentclass[12pt]{article}

% --- Página y tipografía ---
\usepackage[letterpaper,margin=2.5cm]{geometry}
\usepackage[T1]{fontenc}
\usepackage[utf8]{inputenc} % si compilas con pdfLaTeX
\usepackage{lmodern}
\usepackage{microtype}

% --- Imágenes y color ---
\usepackage{graphicx}
\usepackage{xcolor}

% --- Control fino de espacios ---
\usepackage{setspace}
\setlength{\parindent}{0pt}

\begin{document}
\thispagestyle{empty}

% ===== Encabezado con logos + texto =====
\begin{minipage}[c]{0.18\textwidth}
    \centering
    % Cambia por tu logo izquierdo
    \includegraphics[width=0.95\linewidth]{img/logo_usac.jpeg}
\end{minipage}
\hfill
\begin{minipage}[c]{0.60\textwidth}
    \small
    Universidad de San Carlos de Guatemala\\
    Escuela de Ciencias Físicas y Matemáticas\\
    Priscila Ruiz Rosal\\
    Carnet: 202606968\\
    Programación 1\\
\end{minipage}
\hfill
\begin{minipage}[c]{0.18\textwidth}
    \centering
    % Cambia por tu logo derecho
    \includegraphics[width=1.4\linewidth]{img/logo_ecfm.jpg}
\end{minipage}

\vspace{0.5cm}

% Línea horizontal superior (gruesa)
\noindent\rule{\textwidth}{1.2pt}

\vspace{0.2cm}

% ===== Título =====
\begin{center}
    {\Large\scshape Tarea 1}\\[0.3em]
\end{center}

\vspace{0.1cm}

% Fecha
\begin{center}
    \small\scshape 6 de febrero de 2026
\end{center}

\vspace{0.2cm}

% Línea horizontal inferior (gruesa)
\noindent\rule{\textwidth}{1.2pt}

\vspace{0.6cm}

% ===== Caja de instrucciones =====
\noindent
\colorbox{gray!35}{%
    \parbox{\textwidth}{%
        \vspace{0.6em}
        \textbf{Instrucciones}\\[0.3em]
        \small {Redacta un breve ensayo en el que describas cuál sería tu área de investigación 
        en Física o Matemáticas y explica por qué la programación te sería útil en dicho campo.}

        \vspace{0.8em}
    }%
}
\\
\\
\\
\\
Mi interés por las matemáticas como licenciatura universitaria es relativamente nuevo.
El año pasado noté que muchos de mis gustos y curiosidades convergían en ella y decidí explorarla como carrera.
Eso me trajo a la Escuela de Ciencias Físicas y Matemáticas (ECFM) de la Universidad San Carlos de Guatemala (USAC).
Aquí, a través de consultas sobre el pénsum, vine a confirmar que \textbf{Matemática Aplicada} satisfacía tanto mis nichos de interés, como otorgarme gran versatilidad de aplicación.
\\
\\
El programa de estudios contiene 25 cursos electivos.
Para cerrar pénsum se requiere cursar al menos ocho de ellos.
Estos, a su vez, se pueden agrupar en nueve enfoques de estudios:
\\
\begin{itemize}
    \item ciencias de la computación,
    \item geometría y álgebra modernas,
    \item estadística y modelación,
    \item física-matemática,
    \item economía y actuariado,
    \item ecuaciones diferenciales,
    \item ciencia de datos,
    \item análisis matemático y
    \item docencia e investigación pura.
\end{itemize}

Desde mi punto de vista, todas y cualquiera de estas áreas se pueden beneficiar de la utilidad de saber programación.
Ciencias de la computación y ciencia de datos son las que están más obviamente relacionadas.
Sin embargo, el modelado de datos, la utilización de programas híper específicos para ciertos campos o siquiera poder comprender el trabajo realizado por los encargados de informática en un equipo multidisciplinario, ya son razones suficientes.
\\
\\
Aprender de programación implica conocer una herramienta más al servicio de cualquier fin en matemáticas, incluso si esto se resuma a saber que todavía no existen las capacidades de explorar un tema que igualmente es demasiado extenso para hacerlo a mano.
\\
\\
Por todo lo anterior, si yo tuviera que elegir un campo de investigación en el cual aplicar la programación, no podría.
Tengo la certeza de que todo lo que aprenda me será útil, pero al mismo tiempo sé que el mismo estudio de la carrera va a desarrollar en mí intereses más específicos que ahora, en primer año, me son incógnitos.


\end{document}

\end{document}
